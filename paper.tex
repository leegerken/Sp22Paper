\documentclass{article}
\usepackage[margin=1.0in]{geometry}

\title{Language Acquisition Theory and its Compatibility with the Classroom Setting}
\author{Lee Gerken @ Bowling Green State University}
\date{April 2022}


\begin{document}
\maketitle
\begin{abstract}
This paper aims to map out the discourse of language acquisition theory and its compatibility with the classroom setting.
In the second section I will briefly describe the research methods I used to verify L2 language acquisition theories with primary sources.
The third section will attempt to summarize the procession of language acquistion theories within the field of linguistics in order to familarize the reader with the discourse and the path it has taken.
In the fourth section we will do the same from the perspective of educators and the application of these theories; 
The fifth section will focus on the compatibility between the language acquisiton theories presented and the classroom setting.
\end{abstract}


\section{Introduction}

\section{Research Methods}

\section{Language Acquisition Theory Timeline}

\subsection{Chomsky and Exposure}
\subsection{Krashen and Input}
\subsection{Intake}
\subsection{Output}
\subsection{Summary}
\section{Language Acquisition Theory and the Classroom}
\section{Research findings}
\subsection{Observations in L2 Japanese}
\subsection{SRS shift and online communities}

\end{document}
